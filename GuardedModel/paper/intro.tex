% !TeX root = main.tex
% !TeX spellcheck = en-CA
%%% TeX-command-extra-options: "-shell-escape"

\section{Introduction}

Gradual dependent give principled static and dynamic semantics to programs where part of a type or term is missing.
By allowing imprecision, gradual dependent types allow for smooth migration of code
from non-dependently typed languages, or even untyped languages, to full dependent types,
allowing the programmer run and test their code even when the full type details haven't been figured out.
% Likewise, the common practice of ``programming by holes''~\citep{idrisBook} is enhanced:
% code containing holes can still be safely run or tested.
This migration is easiest in languages fulfilling the \textit{gradual guarantees} of
\citet{refinedCriteria}, which state replacing part of a program with $\gqm$ creates no new
static or dynamic type errors. The gradual guarantees ensure that, when an error is encountered,
the problem is never too few types, but that two types in the program are fundamentally incompatible.

However, the benefits of gradual dependent types have not been realized,
since existing developments have enabled the gradual guarantees
at the expense of other desirable properties.
\Citet{Eremondi:2019:ANG:3352468.3341692} presented a dependent calculus supporting the gradual guarantees,
but relied on a termination argument that does not scale to inductive types.
\Citet{bertrand:gcic} presented two extensions of the Calculus of Inductive Constructions (CIC)
that satisfy the gradual guarantees, but one
has undecidable type checking and the other rejects some well-typed static CIC programs.
% The third GCIC variant has decidable type checking and conservatively extends CIC, but does not support
% the gradual guarantees.
They show that, to a degree, these sacrifices are unavoidable, to a degree:
no dependently typed language can satisfy all of
strong normalization, conservative extension of CIC, and the Embedding-Projection Pairs (EP-pairs) property,
a strengthening of the gradual guarantees.


Another obstacle to the adoption of gradual dependent types
is that gradual dependent types have not yet been meaningfully implemented.
Constructing a compiler for a dependently typed language is a massive engineering effort,
and involves writing a type checker, a convertability check for terms, and unification engine for inference,
in addition to the code generation and optimization.
Writing a compiler for a gradual dependently typed language involves all of this work, plus extra
handling to ensure safety in the presence of type imprecision.


% On the theoretical side, the story of gradual dependent types thus far has been one of compromise.
% Several systems have been developed that support limited forms of dependency, such as
% refinement types~\citep{Lehmann:2017:GRT:3009837.3009856,zalewskilambdadb}
% or label-dependent types~\citep{10.1145/3485485}
% \Citet{Eremondi:2019:ANG:3352468.3341692} presented GDTL,
% a foundational calculus with full dependent types.
% GDTL features
% decidable type checking,
% and allowed the imprecise term $\gqm$ to replace any part of a term or type.
% However, the termination argument for GDTL does not support inductive types.
% The most comprehensive development of gradual dependent types is GCIC~\citep{bertrand:gcic},
% which extends the Calculus of Inductive Constructions (CIC) with gradual types.
% GCIC comes in three, each of which satisfies two of the following properties:
% (1) decidable type checking, (2) the gradual guarantees, and (3) conservatively extending CIC.
% \je{TODO: mention equality if either paper gets accepted}
% The authors show that, to a degree, the compromise of GCIC is necessary:
% they establish a ``fire-triangle'' theorem that shows no dependently typed language can satisfy
% strong normalization, conservative extension of CIC, and the Embedding-Projection Pairs (EP-pairs) property,
% a strengthening of the gradual guarantees.


% In this paper, we take the view that the fire triangle properties are means to an end,
% rather than ends unto themselves. Conservatively extending CIC is a useful trait on its own,
% since it ensures that no ill-typed static programs are allowed in a gradual language.
% Strong normalization, conversely, is mainly useful to
% prove that type-checking is decidable, as well as to establish logical consistency.
% Likewise, EP-pairs are mainly useful for showing the gradual guarantees, and showing that
% gradual programs to not unnecessarily produce $\gqm$ as their result.

We address both these shortcomings in \lang, a \underline{Gr}adual language with \underline{Ind}uctive types.
\lang sacrifices
strong normalization and EP-pairs, but keeps the properties that we actually want: decidable type checking,
(weak) consistency and canonicity, the gradual guarantees, and conservatively extending CIC.
Because type checking is decidable, \lang can be translated into the core calculi of existing dependently typed compilers.
Moreover, we show that \lang does not violate static reasoning principles: propositionally-equal
CIC terms embedded in \lang are observationally-equivalent,
and casts only change the error-behavior of terms, not the concrete results produced.
% and while we lack full EP-pairs,
% we show that casts only change the behaviour of terms relative to dynamic type errors:
% applying casts may produce a term that raises an error in more or less situations, but when a concrete
% result is produced, it is the same result.
% \je{TODO explain EP pairs}

Our main contribution is a translation from \lang to a static type theory:
\begin{itemize}
          \item For implementation, the translation means that existing normalizers
                and code generators can be used ``off-the-shelf'' to compile \lang programs;
  \item For metatheory, the translation serves as a syntactic model in the style of \citet{10.1145/3018610.3018620}, which we use to prove the gradual guarantees
        and other metatheoretic properties;
  \item To enable decidable type checking, we adapt approximate normalization
        from \citet{Eremondi:2019:ANG:3352468.3341692} to a cast calculus, using the syntactic
        model to prove termination;
  \item To model run-time semantics, we translate to \textit{guarded type theory}~\citep{TODO},
        whose non-positive recursive types allow the non-termination of gradual types to be
        exactly represented in a consistent target language;
  \item  Our translation and the theorems about it have been mechanized in Guarded Cubical Agda~\citep{TODO}
\end{itemize}

\section{Two Problems, One Solution}


\subsection{Translation to Support Implementation}

\subsubsection{Don't Reinvent The Wheel}

\subsubsection{An Implementation Strategy}

\subsubsection{The Idris Model of Non-Termination}

\subsubsection{Translating Approximate Normalization}


\subsection{Metatheory}

\subsubsection{Extinguishing the Fire Triangle}

\subsubsection{Static Reasoning in Gradual Code}

\subsection{Modelling Gradual Dependent Types}


\subsubsection{Modelling Approximate Normalization}

\subsubsection{Guarded Type Theory}

\subsubsection{Relating Approximate and Exact Normalization}
