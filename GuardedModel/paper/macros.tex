%!TeX root = main.tex
%!TeX spellcheck = en-US

%Renew this in appendix to have some things that only show up when duplicated in the appendix
\newcommand{\ifapx}[1]{}
\newcommand{\ifnotapx}[1]{#1}
%Maybe show rule names
\newcommand{\mname}[1]{\ifapx{#1}}

\usepackage{tikz-cd}
% \usetikzlibrary{external}
% \tikzexternalize

%William Bowman's Smallify.tex
%https://gist.github.com/wilbowma/545b0a315667f41ae2bcd0f3e8a32b95
\usepackage{letltxmacro}

\LetLtxMacro{\oldfigure}{\figure}
\LetLtxMacro{\oldendfigure}{\endfigure}

\LetLtxMacro{\oldcaption}{\caption}

\renewenvironment{figure}
{\oldfigure}
{\vspace{-2ex}\oldendfigure}

\renewcommand{\caption}[1]{\vspace{-0.25ex}\vspace{-\baselineskip}\oldcaption{#1}}

\renewcommand{\MathparLineskip}{\lineskiplimit=.4\baselineskip\lineskip=.4\baselineskip plus .2\baselineskip}

\newcommand{\ifapxCaption}[1]{\caption{#1}}

%% Proofs at the end
\usepackage[createShortEnv]{proof-at-the-end}
% \newtheorem{thm}{Theorem}[section]
% \newtheorem*{thm*}{Theorem}
% \providecommand*\thmautorefname{Theorem}
% % Lemmata
% \newtheorem{lemma}[thm]{Lemma}
% \newtheorem*{lemma*}{Lemma}
% \providecommand*\lemmaautorefname{Lemma}

\include{proofSetting}

\usepackage{iftex}
\ifPDFTeX
\usepackage[utf8]{inputenc}
\usepackage[inline]{enumitem}
\DeclareUnicodeCharacter{2029}{}
\fi

% \usepackage[inline]{enumitem}


\usepackage{underscore}



\usepackage[nomain,
            order=word,
            hyperfirst=false,
            acronym,
            shortcuts,
            nonumberlist]{glossaries}


% \usepackage{idrislang}
% \usepackage{colour-blind}






\newcommand{\ttimes}{\staticstyle{\widetilde{\times}}}



\newcommand{\natInst}{\stt{NatInst}}
\newcommand{\vecInst}{\stt{VecInst}}
\newcommand{\eqInst}{\stt{VecInst}}
\newcommand{\typeInst}[1]{{\stt{IType}_{\staticstyle{#1}}}}

% \newcommand{\ix}[1]{\color{black}{\mathit{#1}}}

% \NewDocumentCommand{\qm}{m}{{\gradualstyle{\textbf{?}_{#1}}}}
\newcommand{\TT}{$\mathsf{TT}$}
% \NewDocumentCommand{\Ty}{m}{{\llparenthesis #1 \rrparenthesis}  }
% \NewDocumentCommand{\Tm}{mm}{{{\llbracket #2 \rrbracket}_{#1}} }
% \NewDocumentCommand{\codeMu}{m}{ {\mu  \IfSubStr{#1}{ }{(#1)}{#1}} }
%\IfSubStr{#1}{ }{(#1)}{#1}
% \NewDocumentCommand{\codeMuu}{m}{ {\tilde{\mu} \llbracket #1 \rrbracket } }
% \NewDocumentCommand{\sType}{m}{{\staticstyle{\mathbf{Type}_{#1}}}}
% \NewDocumentCommand{\sGType}{m}{{\staticstyle{\mathbf{GType}_{#1}}}}
% \NewDocumentCommand{\asType}{m}{{\staticstyle{#1_{\sim}}}}

% \NewDocumentCommand{\gType}{m}{{\gradualstyle{\mathbf{Type}_{#1}}}}
% \NewDocumentCommand{\Dyn}{m}{{\staticstyle{\mathbb{U}^?_{#1}}}}
% \NewDocumentCommand{\toDyn}{mm}{{\stt{toDyn}_{#1}\ #2}}
% \NewDocumentCommand{\fromDyn}{mm}{{\stt{fromDyn}_{#1}^{#2}}}
%
% \defineStaticFun{Opt}
% \defineStaticFun{Some}
% \defineStaticFun{sSucc}{S}
% \defineStaticFun{sCons}{Cons}

% \newcommand{\default}[1]{{\staticstyle{\emptyset}_{#1}}}
% \newcommand{\None}{\staticstyle{\mathtt{None}}}
% \NewDocumentCommand{\Code}{m}{{\mathtt{Code}_{#1}}}
% \newcommand{\unit}{{\mathbf{1}}}
% \NewDocumentCommand{\unitVal}{}{()}
% \NewDocumentCommand{\seq}{m}{{\overrightarrow{#1}}}
% \NewDocumentCommand{\codeType}{m}{{\llbracket{#1}\rrbracket}}
% \NewDocumentCommand{\inMu}{m}{{\langle #1 \rangle}}
% \newcommand{\Jt}{\tilde{J}}

\newcommand{\app}{\staticstyle{(\$)}}

% \NewDocumentCommand{\Nat}{}{\stt{Nat}}
\NewDocumentCommand{\Zero}{}{\stt{Z}}
\NewDocumentCommand{\gSucc}{g}{\gradualstyle{\gtt{S}\IfValueT{#1}{\ #1}}}
% \NewDocumentCommand{\sSucc}{g}{\staticstyle{\stt{S}\IfValueT{#1}{\ #1}}}

\NewDocumentCommand{\gNil}{g}{\gradualstyle{{Nil}\IfValueT{#1}{\ #1}}}
\NewDocumentCommand{\sNil}{g}{\staticstyle{{Nil}\IfValueT{#1}{\ #1}}}

\NewDocumentCommand{\gCons}{gggg}{\gradualstyle{\gtt{Cons}\IfValueT{#1}{\ #1}\ \IfValueT{#2}{\ #2}\ \IfValueT{#3}{\ #3}\ \IfValueT{#4}{\ #4}}}
% \NewDocumentCommand{\sCons}{gggg}{\staticstyle{\stt{Cons}\IfValueT{#1}{\ #1}\ \IfValueT{#2}{\ #2}\ \IfValueT{#3}{\ #3}\ \IfValueT{#4}{\ #4}}}

\NewDocumentCommand{\gVec}{gg}{\gradualstyle{{Vec}\IfValueT{#1}{\ #1}\ \IfValueT{#2}{\ #2}}}
\NewDocumentCommand{\sVec}{gg}{\staticstyle{{Vec}\IfValueT{#1}{\ #1}\ \IfValueT{#2}{\ #2}}}

% \NewDocumentCommand{\VVec}{}{\stt{Vec}}
% \NewDocumentCommand{\Nil}{}{\stt{Nil}}
% \NewDocumentCommand{\Cons}{}{\stt{Cons}}
% \NewDocumentCommand{\Refl}{}{\stt{Refl}}
% \NewDocumentCommand{\gRefl}{gg}{\gradualstyle{\gtt{Refl}\IfValueT{#1}{\ #1}\IfValueT{#2}{\ #2}}}
% \NewDocumentCommand{\sRefl}{gg}{\staticlstyle{\stt{Refl}\IfValueT{#1}{\ #1}\IfValueT{#2}{\ #2}}}

\NewDocumentCommand{\conv}{gg}{\staticstyle{\stt{conv}\IfValueT{#1}{\ #1}\IfValueT{#2}{\ #2}}}

\NewDocumentCommand{\gfst}{m}{\gradualstyle{\pi_1}{#1}}
\NewDocumentCommand{\gsnd}{m}{\gradualstyle{\pi_2}{#1}
\NewDocumentCommand{\gthd}{m}{\gradualstyle{\pi_3}{#1}}}
\NewDocumentCommand{\sfst}{m}{\staticstyle{\pi_1}{#1}}
\NewDocumentCommand{\ssnd}{m}{\staticstyle{\pi_2}{#1}
\NewDocumentCommand{\sthd}{m}{\staticstyle{\pi_3}{#1}}}
\NewDocumentCommand{\opt}{}{\staticstyle{\stt{option}}}
\NewDocumentCommand{\trans}{}{\staticstyle{\stt{trans}}}

\NewDocumentCommand{\optapp}{mm}{\staticstyle{\stt{optapp}\ #1\ #2}}

\NewDocumentCommand{\cod}{m}{\stt{cod}#1}
\NewDocumentCommand{\dom}{m}{\stt{dom}#1}

% \NewDocumentCommand{\PC}{gmgm}{\staticstyle{#2 \mathrel{{}_{#1}{\cong}_{#3}} #4}}



% \mathlig{~->}{\widetilde\to}

\usepackage{lstautogobble}
\usepackage{listings}

\lstdefinelanguage{Agda}%
  {morekeywords={let,in,as,data,record,import,infix,infixl,infixr,module,open,renaming,using,where,\_},%
   morekeywords=[2]{Set,Set1,Set2,Type},%
  literate=*%
     {?}{$\mathrm{\gqm}$}1
     {->}{$\mathrm{\to}$}2,
   otherkeywords={=,:,(,),\{,\},:=,;},
   sensitive=true,%
   morecomment=[n]{\{-}{-\}},%
   morecomment=[l]{--},%
   morestring=[b]{"}%
  }[keywords,comments,strings]%


\lstnewenvironment{gradualCode}{\lstset{language=Agda,
    basicstyle=\rmfamily\color{RoyalBlue},
    columns=flexible, mathescape=true}}{}

\lstnewenvironment{agdaCode}{\lstset{language=Agda,
    basicstyle=\rmfamily\color{Black},
    columns=flexible, mathescape=true}}{}

\lstnewenvironment{staticCode}{\lstset{language=Agda,
    basicstyle=\sffamily\color{BrickRed},
        literate=*%
     {?}{$\mathrm{\gqm}$}1
     {Type}{$\mathcal{U}$ }1
     {->}{$\mathrm{\to}$}2,
     columns=flexible, mathescape=true}}{}

\renewcommand{\tr}{\staticstyle{tr}}

\newcommand{\gline}[1]{
 \lstinline[language=Agda, columns=flexible, basicstyle=\color{RoyalBlue}, mathescape]{#1}
 }
\newcommand{\agdaline}[1]{
 \lstinline[language=Agda, columns=flexible, basicstyle=\color{Black}, mathescape]{#1}
 }

 \newcommand{\sline}[1]{
 \lstinline[language=Agda,
     literate=*%
i     {?}{$\mathrm{\gqm}$}1
     {Type}{$\mathcal{U}$ }1
     {->}{$\mathrm{\to}$}2,
     columns=flexible, basicstyle=\sffamily\color{BrickRed}, mathescape]{#1}
 }
 % \lstMakeShortInline[language=Agda,basicstyle=\small\ttfamily]|

\newcommand{\Code}{\staticstyle{\mathbf{Code}}}
\newcommand{\El}{\staticstyle{\mathbf{El}}}
\newcommand{\runCode}{\staticstyle{\mathbf{Code}^{run}}}
\newcommand{\runEl}{\staticstyle{\mathbf{El}^{run}}}

% \renewcommand{\sType}[1]{\s{\mathbf{Type}}


% \newcommand{\E}[2]{{\color{black}\mathcal{E} \llbracket} \g{ #1 } {\color{black}\rrbracket_{\g{ #2 }}} }
% \newcommand{\Erun}[2]{{\color{black}\mathcal{E}^{run} \llbracket} \g{ #1 } {\color{black}\rrbracket_{\g{ #2 }}} }
\newcommand{\T}[1]{ {\color{black}\mathcal{T}\llbracket} \g{ #1 } {\color{black}\rrbracket}}
% \newcommand{\Trun}[1]{ {\color{black}\mathcal{T}^{run}\llbracket} \g{ #1 } {\color{black}\rrbracket}}
% \renewcommand{\C}[1]{{\color{black}\mathcal{C}\llbracket} \g{ #1 } {\color{black}\rrbracket}}
% \newcommand{\V}[2]{{\color{black}\mathcal{V}\llbracket} \s{ #1 } {\color{black}\rrbracket_{\g{ #2 }}} }
\newcommand{\V}[1]{{\color{black}\mathcal{V}\llbracket} \s{ #1 } {\color{black}\rrbracket} }
% \newcommand{\N}[2]{{\color{black}\mathcal{N}\llbracket} \s{ #1 } {\color{black}\rrbracket_{\g{ #2 }}} }


\newcommand{\R}[2]{{\color{black}\mathcal{R}}^{\s{ #1 }}\s{#2}}

\usepackage{sansmath}


\newcommand{\lang}{\ensuremath{\sf{GEq}}\xspace}
\newcommand{\clang}{\ensuremath{\sf{CastEq}}\xspace}
% \newcommand{\genlang}{\ensuremath{\sf{GEq}}\xspace}
% \newcommand{\synlang}{\ensuremath{\genlang_{Syn}}\xspace}
% \newcommand{\semlang}{\ensuremath{\genlang_{Sem}}\xspace}
% \newcommand{\alglang}{\ensuremath{\genlang_{Alg}}\xspace}

% \newcommand{\syn}{{\sf{GEqSyn}}\xspace}
% \newcommand{\sem}{{\sf{GEqSem}}\xspace}
% \newcommand{\alg}{{\sf{GEqAlg}}\xspace}

% \newcommand{\slang}{{\sf{StIf}}\xspace}
% \newcommand{\slangDesc}{\underline{St}atic language with \underline{If}-else branching\xspace}
% \newcommand{\silang}{{\sf{StInd}}\xspace}
% \newcommand{\selang}{{\sf{StEq}}\xspace}

% \newcommand{\clang}{{\sf{CastIf}}\xspace}
% \newcommand{\clangDesc}{\underline{Cast} calculus with \underline{If}-else branching\xspace}
% \newcommand{\cilang}{{\sf{CastInd}}\xspace}
% \newcommand{\celang}{{\sf{CastEq}}\xspace}

% \newcommand{\tlang}{{\sf{StIR}}\xspace}
% \newcommand{\tlangDesc}{\underline{Static} language with \underline{I}nduction-\underline{Recursion}\xspace}
% % \newcommand{\gtt}{\textsc{SDIndGTT}\xspace}
% % \newcommand{\rlang}{\textsc{CastRun}\xspace}

\newcommand{\cast}[2]{\g{\langle #2 <= #1 \rangle}}
\newcommand{\castnog}[2]{\g{\langle} #2 \g{<=} #1 \g{\rangle}}
\newcommand{\castenv}[2]{{\langle} #2 {<=} #1 {\rangle}}
% \newcommand{\rmeet}[3]{\g{#1 \sqcap_{#3} #2 }}
% \newcommand{\rmeetnog}[3]{{#1 \g{\sqcap}_{#3} #2}}
\newcommand{\grefl}[3]{\g{refl_{#1 |- #2 \cong  #3}}}
\newcommand{\greflnog}[3]{\g{refl}_{#1 \g{|-} #2 \g{\cong} #3}}

% % \newcommand{\upCast}[2]{\g{\langle #2 \nwarrow #1 \rangle}}
% \newcommand{\downCast}[2]{\g{\langle #2 \swarrow  #1 \rangle}}
% \newcommand{\genCast}[2]{\g{\langle #2 \mathrlap{\,\nwarrow}\swarrow  #1 \rangle}}
%plainly styled cast
\newcommand{\pcast}[2]{\g\langle #2 \g{<=} #1 \g\rangle}
% \newcommand{\ev}[1]{\g{\langle #1 \rangle}}

% \newcommand{\sTop}{\s{\textsf{\bf ?} }}
% \newcommand{\sBot}{\s{\sansmath \mho} }
\newcommand{\iswf}{\s{{iswf}}}
\newcommand{\WF}{\s{{WF}}}
\newcommand{\field}{\s{\textit{field}}}
\newcommand{\sSigma}{{\color{BrickRed}\mathrm\Sigma}}
\newcommand{\sPi}{{\color{BrickRed}\mathrm\Pi}}
\newcommand{\uttrans}[1]{{\llbracket \g{#1} \rrbracket}}
\newcommand{\interpCode}[1]{\s{\llparenthesis #1\rrparenthesis_\Code}}
\newcommand{\interp}[1]{\s{\llparenthesis #1 \rrparenthesis}}

\newcommand{\sind}[2]{\s{\mathsf{ind}_{#1}(}#2\s{)}}
\newcommand{\gind}[2]{\g{\mathsf{ind}_{#1}(}#2\g{)}}
\newcommand{\selim}{\s{\mathsf{elim}}}
\newcommand{\gelim}{\g{\mathsf{elim}}}


\newcommand{\varAE}{{\!\! \textit{\AE} }}

\newcommand{\elabsto}{\rightarrowtriangle}
\newcommand{\echeck}[3]{#1 \elabsto #3 <= #2}
\newcommand{\esynth}[3]{#1 \elabsto #3 => #2}
\newcommand{\epsynth}[4]{#2 \elabsto #4 =>_{#1} #3}
\newcommand{\etype}[3]{#1 \elabsto #3 : \gType{}_{=>\g{#2}} }
\newcommand{\redsto}{\leadsto}

\newcommand{\DCat}[1]{{\g{D^C\langle #1\rangle}}}
\newcommand{\DCatnog}[1]{{\g{D^C\langle} #1\g{\rangle}}}
% \newcommand{\seqSqube}[1]{\sqube_{{#1}}}
\newcommand{\sqube}{\sqsubseteq}
\newcommand{\squbr}{\sqsubseteq_{\mathsf{Surf}}}
% \newcommand{\squbo}{\sqsubseteq_{obs}}
\newcommand{\squbo}{\sqsubseteq^{\Vdash}}
\newcommand{\squbB}{\sqsubseteq_{\bB}}
\newcommand{\squbG}{\sqsubseteq^{Ctx}}
\newcommand{\squbstar}{\sqsubseteq^{C*}}
\newcommand{\sqeqstar}{\sqsupseteq\sqsubseteq^{C*}}
\newcommand{\squbs}{\sqsubseteq_{\alpha}}
% \newcommand{\squbstep}{\sqube_{\leadsto}}
\newcommand{\equiprecstep}{\sqsupseteq\sqube_{\leadsto}}
\newcommand{\qmt}[1]{\gqm_{\g{#1}}}
\newcommand{\qml}{\gqm_{\gType{\ell}}}
\newcommand{\qmat}[1]{\g{\gqm_{#1}}}
\newcommand{\attagl}[2]{\g{\langle#1\rangle_{\g\ell}#2}}

% \newcommand{\sqm}


% Cleverref config
\crefformat{section}{\S#2#1#3}
\crefformat{subsection}{\S#2#1#3}
\crefformat{subsubsection}{\S#2#1#3}
\crefrangeformat{section}{\S\S#3#1#4 to~#5#2#6}
\crefmultiformat{section}{\S\S#2#1#3}{ and~#2#1#3}{, #2#1#3}{ and~#2#1#3}
\Crefformat{section}{Section #2#1#3}
\Crefformat{subsection}{Section #2#1#3}
\Crefformat{subsubsection}{Section #2#1#3}
\Crefrangeformat{section}{Sections #3#1#4 to~#5#2#6}
\Crefmultiformat{section}{Sections #2#1#3}{ and~#2#1#3}{, #2#1#3}{ and~#2#1#3}

\newcommand{\tagOf}{\mathit{tagOf}}
\newcommand{\typeTagOf}{\mathit{typeTagOf}}
\newcommand{\levelOne}{{\mathcal{L}^1}}

% \newcommand\psynth{{=>}^{*}}




\renewcommand{\slambda}[3]{\s{\lambda(#1 : #2)\ldotp #3}}
\renewcommand{\rlambda}[3]{\r{\lambda(#1 : #2)\ldotp #3}}
\renewcommand{\glambda}[3]{\g{\lambda(#1 : #2)\ldotp #3}}



\newcommand{\jform}[1]{\fbox{#1}\hspace{\fill}\\}
\newcommand{\jformlow}[1]{\fbox{#1}\hspace{\fill}\vspace{-1em}\\}
% \newcommand{\synth}{\Rightarrow}
\newcommand{\psynth}[1]{{\,\mathbin{\Rightarrow_{#1}}}\,}
\newcommand{\prepsynth}[1]{{\,\mathbin{\Rightarrow*_{#1}}}\,}
\newcommand{\psynthstar}[1]{{\,\mathbin{\Rightarrow^{*}_{#1}}}\,}
% \newcommand{\check}{\Leftarrow}
\newcommand{\ssorts}{\sType{ }}
% \newcommand{\ulev}[1]{^{@#1}}
\newcommand{\ind}{\mathtt{ind}}
\newcommand{\smatch}[4]{\s{\operatorname{\ind}_{#1}(#2,#3,#4)}}
\newcommand{\smatchnoarg}[1]{\s{\operatorname{\ind}_{#1}}}
\newcommand{\rmatch}[4]{\r{\operatorname{\ind}_{#1}(#2,#3,#4)}}
\newcommand{\gmatch}[4]{\g{\operatorname{\ind}_{#1}(#2,#3,#4)}}
\newcommand{\gmatchNoarg}[1]{\g{\operatorname{\ind}_{#1}}}

\newcommand{\fix}[3]{\operatorname{\mathtt{fix}}#1 : #2 := #3}
\newcommand{\pars}{\operatorname{\mathbf{\color{black} Params}}}
\newcommand{\indices}{\operatorname{\mathbf{\color{black} Indices}}}
\newcommand{\ivals}[3]{\operatorname{\mathbf{\color{black} IValsFor}}(#1,#2,#3)}
\newcommand{\iargs}[4]{\operatorname{\mathbf{\color{black} IArgsFor_{#1}}}(#2,#3,#4)}
\newcommand{\args}{\operatorname{\mathbf{\color{black} Args}}}
\newcommand{\parsub}[1]{[#1]}
\newcommand{\hlev}[1]{_{#1}}
\newcommand{\ulev}[1]{\scalebox{0.7}{@\{#1\}}}


\newcommand{\J}{\mathbf{J}}
\newcommand{\Kax}{\mathbf{K}}

\newcommand{\defprec}{{\sqube_{\stepsto}}}
\newcommand{\defsuprec}{{\sqube^{\longleftarrow}_{\stepsto}}}
\newcommand{\defcst}{{{\cong}_{\stepsto}}}
\newcommand{\acst}{{{\cong}_{\alpha}}}

\newcommand{\genprec}{\Gbox{\defprec}}
\newcommand{\gensuprec}{\Gbox{\defsuprec}}
\newcommand{\gencst}{\Gbox{\defcst}}

% \newcommand{\gcomp}[1]{\g{ \sqcap_{#1}} }
\usepackage{scalerel}
\DeclareMathOperator*{\bigamp}{\mathlarger{\&}}
\newcommand{\gcomp}[1]{\mathbin{\g{\&_{#1}}}}
\newcommand{\gcompop}{\g{\&}}
% \newcommand{\gennbot}{\Gbox{\not\sqsubseteq\!\!\mho}}
\usepackage{ifsym}

% \newcommand{\proofappendix}{\cref{apx:proofs}}
\newcommand{\proofappendix}{the appendix of the anonymized supplementary material}
% \newcommand{\Proofappendix}{\Cref{apx:proofs}}


% \newcommand{\ruleappendix}{\cref{apx:rules}}
\newcommand{\ruleappendix}{the appendix of the anonymized supplementary material}
% \newcommand{\Ruleappendix}{\Cref{apx:rules}}

\newcommand{\itercomp}{\seq{\bigamp}}

\newcommand{\germ}{\mathsf{\color{black} germ}}
\newcommand{\head}{\mathsf{\color{black} head}}


%%%%%%%%%%%% Just so we can have long lemma lists
% \let\proof\relax
% \let\endproof\relax
% \usepackage{amsthm} %http://ctan.org/pkg/amsthm
% \newtheorem{theorem}{Theorem}
% \newtheoremstyle{exampstyle}
%   {\topsep} % Space above
%   {\topsep} % Space below
%   {} % Body font
%   {} % Indent amount
%   {\bfseries} % Theorem head font
%   {.} % Punctuation after theorem head
%   {.5em} % Space after theorem head
%   {} % Theorem head spec (can be left empty, meaning `normal')
% \theoremstyle{exampstyle} \newtheorem{example}{Example}
% \theoremstyle{exampstyle} \newtheorem{remark}{Remark}
% \theoremstyle{exampstyle} \newtheorem{definition}{Definition}
% \theoremstyle{exampstyle} \newtheorem{lemma}{Lemma}
