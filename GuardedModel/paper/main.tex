% !TeX root = main.tex
% !TeX spellcheck = en-US
%%% TeX-command-extra-options: "-shell-escape"

%%
%% This is file `sample-acmsmall.tex',
%% generated with the docstrip utility.
%%
%% The original source files were:
%%
%% samples.dtx  (with options: `acmsmall')
%%
%% IMPORTANT NOTICE:
%%
%% For the copyright see the source file.
%%
%% Any modified versions of this file must be renamed
%% with new filenames distinct from sample-acmsmall.tex.
%%
%% For distribution of the original source see the terms
%% for copying and modification in the file samples.dtx.
%%
%% This generated file may be distributed as long as the
%% original source files, as listed above, are part of the
%% same distribution. (The sources need not necessarily be
%% in the same archive or directory.)
%%
%% The first command in your LaTeX source must be the \documentclass command.
% \documentclass[autogobble,dvipsnames,acmsmall,anonymous,review]{acmart}
\documentclass[autogobble,dvipsnames,acmsmall,review,screen,anonymous]{acmart}
\makeatletter


% \def\@testdef #1#2#3{%
% \def\reserved@a{#3}\expandafter \ifx \csname #1@#2\endcsname
% \reserved@a  \else
% \typeout{^^Jlabel #2 changed:^^J%
% \meaning\reserved@a^^J%
% \expandafter\meaning\csname #1@#2\endcsname^^J}%
% \@tempswatrue \fi}



\@ifclasswith{acmart}{review}{
  \newcommand{\reviewmode}{}
  % Draft only options
}{
  %Non-review only options
}

\input{sharedmacros}
%!TeX root = main.tex
%!TeX spellcheck = en-US

%Renew this in appendix to have some things that only show up when duplicated in the appendix
\newcommand{\ifapx}[1]{}
\newcommand{\ifnotapx}[1]{#1}
%Maybe show rule names
\newcommand{\mname}[1]{\ifapx{#1}}

\usepackage{tikz-cd}
% \usetikzlibrary{external}
% \tikzexternalize

%William Bowman's Smallify.tex
%https://gist.github.com/wilbowma/545b0a315667f41ae2bcd0f3e8a32b95
\usepackage{letltxmacro}

\LetLtxMacro{\oldfigure}{\figure}
\LetLtxMacro{\oldendfigure}{\endfigure}

\LetLtxMacro{\oldcaption}{\caption}

\renewenvironment{figure}
{\oldfigure}
{\vspace{-2ex}\oldendfigure}

\renewcommand{\caption}[1]{\vspace{-0.25ex}\vspace{-\baselineskip}\oldcaption{#1}}

\renewcommand{\MathparLineskip}{\lineskiplimit=.4\baselineskip\lineskip=.4\baselineskip plus .2\baselineskip}

\newcommand{\ifapxCaption}[1]{\caption{#1}}

%% Proofs at the end
\usepackage[createShortEnv]{proof-at-the-end}
% \newtheorem{thm}{Theorem}[section]
% \newtheorem*{thm*}{Theorem}
% \providecommand*\thmautorefname{Theorem}
% % Lemmata
% \newtheorem{lemma}[thm]{Lemma}
% \newtheorem*{lemma*}{Lemma}
% \providecommand*\lemmaautorefname{Lemma}

\include{proofSetting}

\usepackage{iftex}
\ifPDFTeX
\usepackage[utf8]{inputenc}
\usepackage[inline]{enumitem}
\DeclareUnicodeCharacter{2029}{}
\fi

% \usepackage[inline]{enumitem}


\usepackage{underscore}



\usepackage[nomain,
            order=word,
            hyperfirst=false,
            acronym,
            shortcuts,
            nonumberlist]{glossaries}


% \usepackage{idrislang}
% \usepackage{colour-blind}






\newcommand{\ttimes}{\staticstyle{\widetilde{\times}}}



\newcommand{\natInst}{\stt{NatInst}}
\newcommand{\vecInst}{\stt{VecInst}}
\newcommand{\eqInst}{\stt{VecInst}}
\newcommand{\typeInst}[1]{{\stt{IType}_{\staticstyle{#1}}}}

% \newcommand{\ix}[1]{\color{black}{\mathit{#1}}}

% \NewDocumentCommand{\qm}{m}{{\gradualstyle{\textbf{?}_{#1}}}}
\newcommand{\TT}{$\mathsf{TT}$}
% \NewDocumentCommand{\Ty}{m}{{\llparenthesis #1 \rrparenthesis}  }
% \NewDocumentCommand{\Tm}{mm}{{{\llbracket #2 \rrbracket}_{#1}} }
% \NewDocumentCommand{\codeMu}{m}{ {\mu  \IfSubStr{#1}{ }{(#1)}{#1}} }
%\IfSubStr{#1}{ }{(#1)}{#1}
% \NewDocumentCommand{\codeMuu}{m}{ {\tilde{\mu} \llbracket #1 \rrbracket } }
% \NewDocumentCommand{\sType}{m}{{\staticstyle{\mathbf{Type}_{#1}}}}
% \NewDocumentCommand{\sGType}{m}{{\staticstyle{\mathbf{GType}_{#1}}}}
% \NewDocumentCommand{\asType}{m}{{\staticstyle{#1_{\sim}}}}

% \NewDocumentCommand{\gType}{m}{{\gradualstyle{\mathbf{Type}_{#1}}}}
% \NewDocumentCommand{\Dyn}{m}{{\staticstyle{\mathbb{U}^?_{#1}}}}
% \NewDocumentCommand{\toDyn}{mm}{{\stt{toDyn}_{#1}\ #2}}
% \NewDocumentCommand{\fromDyn}{mm}{{\stt{fromDyn}_{#1}^{#2}}}
%
% \defineStaticFun{Opt}
% \defineStaticFun{Some}
% \defineStaticFun{sSucc}{S}
% \defineStaticFun{sCons}{Cons}

% \newcommand{\default}[1]{{\staticstyle{\emptyset}_{#1}}}
% \newcommand{\None}{\staticstyle{\mathtt{None}}}
% \NewDocumentCommand{\Code}{m}{{\mathtt{Code}_{#1}}}
% \newcommand{\unit}{{\mathbf{1}}}
% \NewDocumentCommand{\unitVal}{}{()}
% \NewDocumentCommand{\seq}{m}{{\overrightarrow{#1}}}
% \NewDocumentCommand{\codeType}{m}{{\llbracket{#1}\rrbracket}}
% \NewDocumentCommand{\inMu}{m}{{\langle #1 \rangle}}
% \newcommand{\Jt}{\tilde{J}}

\newcommand{\app}{\staticstyle{(\$)}}

% \NewDocumentCommand{\Nat}{}{\stt{Nat}}
\NewDocumentCommand{\Zero}{}{\stt{Z}}
\NewDocumentCommand{\gSucc}{g}{\gradualstyle{\gtt{S}\IfValueT{#1}{\ #1}}}
% \NewDocumentCommand{\sSucc}{g}{\staticstyle{\stt{S}\IfValueT{#1}{\ #1}}}

\NewDocumentCommand{\gNil}{g}{\gradualstyle{{Nil}\IfValueT{#1}{\ #1}}}
\NewDocumentCommand{\sNil}{g}{\staticstyle{{Nil}\IfValueT{#1}{\ #1}}}

\NewDocumentCommand{\gCons}{gggg}{\gradualstyle{\gtt{Cons}\IfValueT{#1}{\ #1}\ \IfValueT{#2}{\ #2}\ \IfValueT{#3}{\ #3}\ \IfValueT{#4}{\ #4}}}
% \NewDocumentCommand{\sCons}{gggg}{\staticstyle{\stt{Cons}\IfValueT{#1}{\ #1}\ \IfValueT{#2}{\ #2}\ \IfValueT{#3}{\ #3}\ \IfValueT{#4}{\ #4}}}

\NewDocumentCommand{\gVec}{gg}{\gradualstyle{{Vec}\IfValueT{#1}{\ #1}\ \IfValueT{#2}{\ #2}}}
\NewDocumentCommand{\sVec}{gg}{\staticstyle{{Vec}\IfValueT{#1}{\ #1}\ \IfValueT{#2}{\ #2}}}

% \NewDocumentCommand{\VVec}{}{\stt{Vec}}
% \NewDocumentCommand{\Nil}{}{\stt{Nil}}
% \NewDocumentCommand{\Cons}{}{\stt{Cons}}
% \NewDocumentCommand{\Refl}{}{\stt{Refl}}
% \NewDocumentCommand{\gRefl}{gg}{\gradualstyle{\gtt{Refl}\IfValueT{#1}{\ #1}\IfValueT{#2}{\ #2}}}
% \NewDocumentCommand{\sRefl}{gg}{\staticlstyle{\stt{Refl}\IfValueT{#1}{\ #1}\IfValueT{#2}{\ #2}}}

\NewDocumentCommand{\conv}{gg}{\staticstyle{\stt{conv}\IfValueT{#1}{\ #1}\IfValueT{#2}{\ #2}}}

\NewDocumentCommand{\gfst}{m}{\gradualstyle{\pi_1}{#1}}
\NewDocumentCommand{\gsnd}{m}{\gradualstyle{\pi_2}{#1}
\NewDocumentCommand{\gthd}{m}{\gradualstyle{\pi_3}{#1}}}
\NewDocumentCommand{\sfst}{m}{\staticstyle{\pi_1}{#1}}
\NewDocumentCommand{\ssnd}{m}{\staticstyle{\pi_2}{#1}
\NewDocumentCommand{\sthd}{m}{\staticstyle{\pi_3}{#1}}}
\NewDocumentCommand{\opt}{}{\staticstyle{\stt{option}}}
\NewDocumentCommand{\trans}{}{\staticstyle{\stt{trans}}}

\NewDocumentCommand{\optapp}{mm}{\staticstyle{\stt{optapp}\ #1\ #2}}

\NewDocumentCommand{\cod}{m}{\stt{cod}#1}
\NewDocumentCommand{\dom}{m}{\stt{dom}#1}

% \NewDocumentCommand{\PC}{gmgm}{\staticstyle{#2 \mathrel{{}_{#1}{\cong}_{#3}} #4}}



% \mathlig{~->}{\widetilde\to}

\usepackage{lstautogobble}
\usepackage{listings}

\lstdefinelanguage{Agda}%
  {morekeywords={let,in,as,data,record,import,infix,infixl,infixr,module,open,renaming,using,where,\_},%
   morekeywords=[2]{Set,Set1,Set2,Type},%
  literate=*%
     {?}{$\mathrm{\gqm}$}1
     {->}{$\mathrm{\to}$}2,
   otherkeywords={=,:,(,),\{,\},:=,;},
   sensitive=true,%
   morecomment=[n]{\{-}{-\}},%
   morecomment=[l]{--},%
   morestring=[b]{"}%
  }[keywords,comments,strings]%


\lstnewenvironment{gradualCode}{\lstset{language=Agda,
    basicstyle=\rmfamily\color{RoyalBlue},
    columns=flexible, mathescape=true}}{}

\lstnewenvironment{agdaCode}{\lstset{language=Agda,
    basicstyle=\rmfamily\color{Black},
    columns=flexible, mathescape=true}}{}

\lstnewenvironment{staticCode}{\lstset{language=Agda,
    basicstyle=\sffamily\color{BrickRed},
        literate=*%
     {?}{$\mathrm{\gqm}$}1
     {Type}{$\mathcal{U}$ }1
     {->}{$\mathrm{\to}$}2,
     columns=flexible, mathescape=true}}{}

\renewcommand{\tr}{\staticstyle{tr}}

\newcommand{\gline}[1]{
 \lstinline[language=Agda, columns=flexible, basicstyle=\color{RoyalBlue}, mathescape]{#1}
 }
\newcommand{\agdaline}[1]{
 \lstinline[language=Agda, columns=flexible, basicstyle=\color{Black}, mathescape]{#1}
 }

 \newcommand{\sline}[1]{
 \lstinline[language=Agda,
     literate=*%
i     {?}{$\mathrm{\gqm}$}1
     {Type}{$\mathcal{U}$ }1
     {->}{$\mathrm{\to}$}2,
     columns=flexible, basicstyle=\sffamily\color{BrickRed}, mathescape]{#1}
 }
 % \lstMakeShortInline[language=Agda,basicstyle=\small\ttfamily]|

\newcommand{\Code}{\staticstyle{\mathbf{Code}}}
\newcommand{\El}{\staticstyle{\mathbf{El}}}
\newcommand{\runCode}{\staticstyle{\mathbf{Code}^{run}}}
\newcommand{\runEl}{\staticstyle{\mathbf{El}^{run}}}

% \renewcommand{\sType}[1]{\s{\mathbf{Type}}


% \newcommand{\E}[2]{{\color{black}\mathcal{E} \llbracket} \g{ #1 } {\color{black}\rrbracket_{\g{ #2 }}} }
% \newcommand{\Erun}[2]{{\color{black}\mathcal{E}^{run} \llbracket} \g{ #1 } {\color{black}\rrbracket_{\g{ #2 }}} }
\newcommand{\T}[1]{ {\color{black}\mathcal{T}\llbracket} \g{ #1 } {\color{black}\rrbracket}}
% \newcommand{\Trun}[1]{ {\color{black}\mathcal{T}^{run}\llbracket} \g{ #1 } {\color{black}\rrbracket}}
% \renewcommand{\C}[1]{{\color{black}\mathcal{C}\llbracket} \g{ #1 } {\color{black}\rrbracket}}
% \newcommand{\V}[2]{{\color{black}\mathcal{V}\llbracket} \s{ #1 } {\color{black}\rrbracket_{\g{ #2 }}} }
\newcommand{\V}[1]{{\color{black}\mathcal{V}\llbracket} \s{ #1 } {\color{black}\rrbracket} }
% \newcommand{\N}[2]{{\color{black}\mathcal{N}\llbracket} \s{ #1 } {\color{black}\rrbracket_{\g{ #2 }}} }


\newcommand{\R}[2]{{\color{black}\mathcal{R}}^{\s{ #1 }}\s{#2}}

\usepackage{sansmath}


\newcommand{\lang}{\ensuremath{\sf{GrInd}}\xspace}
\newcommand{\clang}{\ensuremath{\sf{CastInd}}\xspace}
% \newcommand{\genlang}{\ensuremath{\sf{GEq}}\xspace}
% \newcommand{\synlang}{\ensuremath{\genlang_{Syn}}\xspace}
% \newcommand{\semlang}{\ensuremath{\genlang_{Sem}}\xspace}
% \newcommand{\alglang}{\ensuremath{\genlang_{Alg}}\xspace}

% \newcommand{\syn}{{\sf{GEqSyn}}\xspace}
% \newcommand{\sem}{{\sf{GEqSem}}\xspace}
% \newcommand{\alg}{{\sf{GEqAlg}}\xspace}

% \newcommand{\slang}{{\sf{StIf}}\xspace}
% \newcommand{\slangDesc}{\underline{St}atic language with \underline{If}-else branching\xspace}
% \newcommand{\silang}{{\sf{StInd}}\xspace}
% \newcommand{\selang}{{\sf{StEq}}\xspace}

% \newcommand{\clang}{{\sf{CastIf}}\xspace}
% \newcommand{\clangDesc}{\underline{Cast} calculus with \underline{If}-else branching\xspace}
% \newcommand{\cilang}{{\sf{CastInd}}\xspace}
% \newcommand{\celang}{{\sf{CastEq}}\xspace}

% \newcommand{\tlang}{{\sf{StIR}}\xspace}
% \newcommand{\tlangDesc}{\underline{Static} language with \underline{I}nduction-\underline{Recursion}\xspace}
% % \newcommand{\gtt}{\textsc{SDIndGTT}\xspace}
% % \newcommand{\rlang}{\textsc{CastRun}\xspace}

\newcommand{\cast}[2]{\g{\langle #2 <= #1 \rangle}}
\newcommand{\castnog}[2]{\g{\langle} #2 \g{<=} #1 \g{\rangle}}
\newcommand{\castenv}[2]{{\langle} #2 {<=} #1 {\rangle}}
% \newcommand{\rmeet}[3]{\g{#1 \sqcap_{#3} #2 }}
% \newcommand{\rmeetnog}[3]{{#1 \g{\sqcap}_{#3} #2}}
\newcommand{\grefl}[3]{\g{refl_{#1 |- #2 \cong  #3}}}
\newcommand{\greflnog}[3]{\g{refl}_{#1 \g{|-} #2 \g{\cong} #3}}

% % \newcommand{\upCast}[2]{\g{\langle #2 \nwarrow #1 \rangle}}
% \newcommand{\downCast}[2]{\g{\langle #2 \swarrow  #1 \rangle}}
% \newcommand{\genCast}[2]{\g{\langle #2 \mathrlap{\,\nwarrow}\swarrow  #1 \rangle}}
%plainly styled cast
\newcommand{\pcast}[2]{\g\langle #2 \g{<=} #1 \g\rangle}
% \newcommand{\ev}[1]{\g{\langle #1 \rangle}}

% \newcommand{\sTop}{\s{\textsf{\bf ?} }}
% \newcommand{\sBot}{\s{\sansmath \mho} }
\newcommand{\iswf}{\s{{iswf}}}
\newcommand{\WF}{\s{{WF}}}
\newcommand{\field}{\s{\textit{field}}}
\newcommand{\sSigma}{{\color{BrickRed}\mathrm\Sigma}}
\newcommand{\sPi}{{\color{BrickRed}\mathrm\Pi}}
\newcommand{\uttrans}[1]{{\llbracket \g{#1} \rrbracket}}
\newcommand{\interpCode}[1]{\s{\llparenthesis #1\rrparenthesis_\Code}}
\newcommand{\interp}[1]{\s{\llparenthesis #1 \rrparenthesis}}

\newcommand{\sind}[2]{\s{\mathsf{ind}_{#1}(}#2\s{)}}
\newcommand{\gind}[2]{\g{\mathsf{ind}_{#1}(}#2\g{)}}
\newcommand{\selim}{\s{\mathsf{elim}}}
\newcommand{\gelim}{\g{\mathsf{elim}}}


\newcommand{\varAE}{{\!\! \textit{\AE} }}

\newcommand{\elabsto}{\rightarrowtriangle}
\newcommand{\echeck}[3]{#1 \elabsto #3 <= #2}
\newcommand{\esynth}[3]{#1 \elabsto #3 => #2}
\newcommand{\epsynth}[4]{#2 \elabsto #4 =>_{#1} #3}
\newcommand{\etype}[3]{#1 \elabsto #3 : \gType{}_{=>\g{#2}} }
\newcommand{\redsto}{\leadsto}

\newcommand{\DCat}[1]{{\g{D^C\langle #1\rangle}}}
\newcommand{\DCatnog}[1]{{\g{D^C\langle} #1\g{\rangle}}}
% \newcommand{\seqSqube}[1]{\sqube_{{#1}}}
\newcommand{\sqube}{\sqsubseteq}
\newcommand{\squbr}{\sqsubseteq_{\mathsf{Surf}}}
% \newcommand{\squbo}{\sqsubseteq_{obs}}
\newcommand{\squbo}{\sqsubseteq^{\Vdash}}
\newcommand{\squbB}{\sqsubseteq_{\bB}}
\newcommand{\squbG}{\sqsubseteq^{Ctx}}
\newcommand{\squbstar}{\sqsubseteq^{C*}}
\newcommand{\sqeqstar}{\sqsupseteq\sqsubseteq^{C*}}
\newcommand{\squbs}{\sqsubseteq_{\alpha}}
% \newcommand{\squbstep}{\sqube_{\leadsto}}
\newcommand{\equiprecstep}{\sqsupseteq\sqube_{\leadsto}}
\newcommand{\qmt}[1]{\gqm_{\g{#1}}}
\newcommand{\qml}{\gqm_{\gType{\ell}}}
\newcommand{\qmat}[1]{\g{\gqm_{#1}}}
\newcommand{\attagl}[2]{\g{\langle#1\rangle_{\g\ell}#2}}

% \newcommand{\sqm}


% Cleverref config
\crefformat{section}{\S#2#1#3}
\crefformat{subsection}{\S#2#1#3}
\crefformat{subsubsection}{\S#2#1#3}
\crefrangeformat{section}{\S\S#3#1#4 to~#5#2#6}
\crefmultiformat{section}{\S\S#2#1#3}{ and~#2#1#3}{, #2#1#3}{ and~#2#1#3}
\Crefformat{section}{Section #2#1#3}
\Crefformat{subsection}{Section #2#1#3}
\Crefformat{subsubsection}{Section #2#1#3}
\Crefrangeformat{section}{Sections #3#1#4 to~#5#2#6}
\Crefmultiformat{section}{Sections #2#1#3}{ and~#2#1#3}{, #2#1#3}{ and~#2#1#3}

\newcommand{\tagOf}{\mathit{tagOf}}
\newcommand{\typeTagOf}{\mathit{typeTagOf}}
\newcommand{\levelOne}{{\mathcal{L}^1}}

% \newcommand\psynth{{=>}^{*}}




\renewcommand{\slambda}[3]{\s{\lambda(#1 : #2)\ldotp #3}}
\renewcommand{\rlambda}[3]{\r{\lambda(#1 : #2)\ldotp #3}}
\renewcommand{\glambda}[3]{\g{\lambda(#1 : #2)\ldotp #3}}



\newcommand{\jform}[1]{\fbox{#1}\hspace{\fill}\\}
\newcommand{\jformlow}[1]{\fbox{#1}\hspace{\fill}\vspace{-1em}\\}
% \newcommand{\synth}{\Rightarrow}
\newcommand{\psynth}[1]{{\,\mathbin{\Rightarrow_{#1}}}\,}
\newcommand{\prepsynth}[1]{{\,\mathbin{\Rightarrow*_{#1}}}\,}
\newcommand{\psynthstar}[1]{{\,\mathbin{\Rightarrow^{*}_{#1}}}\,}
% \newcommand{\check}{\Leftarrow}
\newcommand{\ssorts}{\sType{ }}
% \newcommand{\ulev}[1]{^{@#1}}
\newcommand{\ind}{\mathtt{ind}}
\newcommand{\smatch}[4]{\s{\operatorname{\ind}_{#1}(#2,#3,#4)}}
\newcommand{\smatchnoarg}[1]{\s{\operatorname{\ind}_{#1}}}
\newcommand{\rmatch}[4]{\r{\operatorname{\ind}_{#1}(#2,#3,#4)}}
\newcommand{\gmatch}[4]{\g{\operatorname{\ind}_{#1}(#2,#3,#4)}}
\newcommand{\gmatchNoarg}[1]{\g{\operatorname{\ind}_{#1}}}

\newcommand{\fix}[3]{\operatorname{\mathtt{fix}}#1 : #2 := #3}
\newcommand{\pars}{\operatorname{\mathbf{\color{black} Params}}}
\newcommand{\indices}{\operatorname{\mathbf{\color{black} Indices}}}
\newcommand{\ivals}[3]{\operatorname{\mathbf{\color{black} IValsFor}}(#1,#2,#3)}
\newcommand{\iargs}[4]{\operatorname{\mathbf{\color{black} IArgsFor_{#1}}}(#2,#3,#4)}
\newcommand{\args}{\operatorname{\mathbf{\color{black} Args}}}
\newcommand{\parsub}[1]{[#1]}
\newcommand{\hlev}[1]{_{#1}}
\newcommand{\ulev}[1]{\scalebox{0.7}{@\{#1\}}}


\newcommand{\J}{\mathbf{J}}
\newcommand{\Kax}{\mathbf{K}}

\newcommand{\defprec}{{\sqube_{\stepsto}}}
\newcommand{\defsuprec}{{\sqube^{\longleftarrow}_{\stepsto}}}
\newcommand{\defcst}{{{\cong}_{\stepsto}}}
\newcommand{\acst}{{{\cong}_{\alpha}}}

\newcommand{\genprec}{\Gbox{\defprec}}
\newcommand{\gensuprec}{\Gbox{\defsuprec}}
\newcommand{\gencst}{\Gbox{\defcst}}

% \newcommand{\gcomp}[1]{\g{ \sqcap_{#1}} }
\usepackage{scalerel}
\DeclareMathOperator*{\bigamp}{\mathlarger{\&}}
\newcommand{\gcomp}[1]{\mathbin{\g{\&_{#1}}}}
\newcommand{\gcompop}{\g{\&}}
% \newcommand{\gennbot}{\Gbox{\not\sqsubseteq\!\!\mho}}
\usepackage{ifsym}

% \newcommand{\proofappendix}{\cref{apx:proofs}}
\newcommand{\proofappendix}{the appendix of the anonymized supplementary material}
% \newcommand{\Proofappendix}{\Cref{apx:proofs}}


% \newcommand{\ruleappendix}{\cref{apx:rules}}
\newcommand{\ruleappendix}{the appendix of the anonymized supplementary material}
% \newcommand{\Ruleappendix}{\Cref{apx:rules}}

\newcommand{\itercomp}{\seq{\bigamp}}

\newcommand{\germ}{\mathsf{\color{black} germ}}
\newcommand{\head}{\mathsf{\color{black} head}}


%%%%%%%%%%%% Just so we can have long lemma lists
% \let\proof\relax
% \let\endproof\relax
% \usepackage{amsthm} %http://ctan.org/pkg/amsthm
% \newtheorem{theorem}{Theorem}
% \newtheoremstyle{exampstyle}
%   {\topsep} % Space above
%   {\topsep} % Space below
%   {} % Body font
%   {} % Indent amount
%   {\bfseries} % Theorem head font
%   {.} % Punctuation after theorem head
%   {.5em} % Space after theorem head
%   {} % Theorem head spec (can be left empty, meaning `normal')
% \theoremstyle{exampstyle} \newtheorem{example}{Example}
% \theoremstyle{exampstyle} \newtheorem{remark}{Remark}
% \theoremstyle{exampstyle} \newtheorem{definition}{Definition}
% \theoremstyle{exampstyle} \newtheorem{lemma}{Lemma}



\acmYear{2022}

\ifdef{\reviewmode}{
  \settopmatter{printfolios=true,printccs=false,printacmref=false}
  \setcopyright{none}
  \renewcommand\footnotetextcopyrightpermission[1]{}
  % \pagestyle{plain}
  \raggedbottom
}{
\setcopyright{acmcopyright}
\copyrightyear{2021}
\acmDOI{10.1145/1122445.1122456}
}



%%
%% \BibTeX command to typeset BibTeX logo in the docs
\AtBeginDocument{%
  \providecommand\BibTeX{{%
    \normalfont B\kern-0.5em{\scshape i\kern-0.25em b}\kern-0.8em\TeX}}}

%% Rights management information.  This information is sent to you
%% when you complete the rights form.  These commands have SAMPLE
%% values in them; it is your responsibility as an author to replace
%% the commands and values with those provided to you when you
%% complete the rights form.


%%
%% These commands are for a JOURNAL article.
\acmJournal{PACMPL}
\acmVolume{37}
\acmNumber{4}
\acmArticle{111}
\acmMonth{7}

%%
%% Submission ID.
%% Use this when submitting an article to a sponsored event. You'll
%% receive a unique submission ID from the organizers
%% of the event, and this ID should be used as the parameter to this command.
%%\acmSubmissionID{123-A56-BU3}

%%
%% The majority of ACM publications use numbered citations and
%% references.  The command \citestyle{authoryear} switches to the
%% "author year" style.
%%
%% If you are preparing content for an event
%% sponsored by ACM SIGGRAPH, you must use the "author year" style of
%% citations and references.
%% Uncommenting
%% the next command will enable that style.
\citestyle{acmauthoryear}


%%
%% end of the preamble, start of the body of the document source.
\begin{document}

% !TeX root = main.tex
% !TeX spellcheck = en-US

\title{A Guraded Syntactic Model of Gradual Dependent Types}
\subtitle{Translation to Support Implementation and Metatheory}
%%
%% The "author" command and its associated commands are used to define
%% the authors and their affiliations.
%% Of note is the shared affiliation of the first two authors, and the
%% "authornote" and "authornotemark" commands
%% used to denote shared contribution to the research.
% \orcid{nnnn-nnnn-nnnn-nnnn}             %% \orcid is optional

  \author{Joseph Eremondi}
  % \authornote{with author1 note}          %% \authornote is optional;
                                          %% can be repeated if necessary

  \affiliation{
    \department{Department of Computer Science}              %% \department is recommended
    \institution{University of British Columbia}            %% \institution is required
    \country{Canada}                    %% \country is recommended
  }
  \email{{jeremond@cs.ubc.ca}}          %% \email is recommended

    %% Author with two affiliations and emails.
  \author{Ronald Garcia}
  \affiliation{
    \institution{University of British Columbia}            %% \institution is required
    \country{Canada}                    %% \country is recommended
  }
  \email{rxg@cs.ubc.ca}          %% \email is recommended

    %% Author with two affiliations and emails.
    \author{\'{E}ric Tanter}
    \affiliation{
      \department{Computer Science Department (DCC)}              %% \department is recommended
      \institution{University of Chile}            %% \institution is required
      \country{Chile}                    %% \country is recommended
    }
    \email{etanter@dcc.uchile.cl}          %% \email is recommended




%%
%% By default, the full list of authors will be used in the page
%% headers. Often, this list is too long, and will overlap
%% other information printed in the page headers. This command allows
%% the author to define a more concise list
%% of authors' names for this purpose.
\renewcommand{\shortauthors}{Eremondi, Garcia, and Tanter}


%%
%% The "title" command has an optional parameter,
%% allowing the author to define a "short title" to be used in page headers.

%%
%% The abstract is a short summary of the work to be presented in the
%% article.
\begin{abstract}
  % !TeX root = main.tex
% !TeX spellcheck = en-US

TODO write abstract

\end{abstract}

%%
%% The code below is generated by the tool at http://dl.acm.org/ccs.cfm.
%% Please copy and paste the code instead of the example below.
%%
% \begin{CCSXML}
% <ccs2012>
%  <concept>
%   <concept_id>10010520.10010553.10010562</concept_id>
%   <concept_desc>Computer systems organization~Embedded systems</concept_desc>
%   <concept_significance>500</concept_significance>
%  </concept>
%  <concept>
%   <concept_id>10010520.10010575.10010755</concept_id>
%   <concept_desc>Computer systems organization~Redundancy</concept_desc>
%   <concept_significance>300</concept_significance>
%  </concept>
%  <concept>
%   <concept_id>10010520.10010553.10010554</concept_id>
%   <concept_desc>Computer systems organization~Robotics</concept_desc>
%   <concept_significance>100</concept_significance>
%  </concept>
%  <concept>
%   <concept_id>10003033.10003083.10003095</concept_id>
%   <concept_desc>Networks~Network reliability</concept_desc>
%   <concept_significance>100</concept_significance>
%  </concept>
% </ccs2012>
% \end{CCSXML}

\ifdef{\reviewmode}{
%
}{
\ccsdesc[500]{Theory of computation~Type structures}
\ccsdesc[500]{Theory of computation~Program semantics}

%%
%% Keywords. The author(s) should pick words that accurately describe
%% the work being presented. Separate the keywords with commas.
\keywords{dependent types, gradual types, inductive families, propositional equality}
}



%%
%% This command processes the author and affiliation and title
%% information and builds the first part of the formatted document.
\maketitle

% !TeX root = main.tex
% !TeX spellcheck = en-CA
%%% TeX-command-extra-options: "-shell-escape"

% The contents of the paper


% !TeX root = main.tex
% !TeX spellcheck = en-CA
%%% TeX-command-extra-options: "-shell-escape"

\section{Introduction}

Gradual dependent give principled static and dynamic semantics to programs where part of a type or term is missing.
By allowing imprecision, gradual dependent types allow for smooth migration of code
from non-dependently typed languages, or even untyped languages, to full dependent types,
allowing the programmer run and test their code even when the full type details haven't been figured out.
% Likewise, the common practice of ``programming by holes''~\citep{idrisBook} is enhanced:
% code containing holes can still be safely run or tested.
This migration is easiest in languages fulfilling the \textit{gradual guarantees} of
\citet{refinedCriteria}, which state replacing part of a program with $\gqm$ creates no new
static or dynamic type errors. The gradual guarantees ensure that, when an error is encountered,
the problem is never too few types, but that two types in the program are fundamentally incompatible.

However, the benefits of gradual dependent types have not been realized,
since existing developments have enabled the gradual guarantees
at the expense of other desirable properties.
\Citet{Eremondi:2019:ANG:3352468.3341692} presented a dependent calculus supporting the gradual guarantees,
but relied on a termination argument that does not scale to inductive types.
\Citet{bertrand:gcic} presented two extensions of the Calculus of Inductive Constructions (CIC)
that satisfy the gradual guarantees, but one
has undecidable type checking and the other rejects some well-typed static CIC programs.
% The third GCIC variant has decidable type checking and conservatively extends CIC, but does not support
% the gradual guarantees.

\Citet{bertrand:gcic} show that, to a degree, these sacrifices are unavoidable, to a degree:
no dependently typed language can satisfy all of
strong normalization, conservative extension of CIC, and the Embedding-Projection Pairs (EP-pairs) property,
a strengthening of the gradual guarantees.


Another obstacle to the adoption of gradual dependent types
is that gradual dependent types have not yet been meaningfully implemented.
Constructing a compiler for a dependently typed language is a massive engineering effort,
and involves writing a type checker, a convertability check for terms, and unification engine for inference,
in addition to the code generation and optimization.
Writing a compiler for a gradual dependently typed language involves all of this work, plus extra
handling to ensure safety in the presence of type imprecision.


% On the theoretical side, the story of gradual dependent types thus far has been one of compromise.
% Several systems have been developed that support limited forms of dependency, such as
% refinement types~\citep{Lehmann:2017:GRT:3009837.3009856,zalewskilambdadb}
% or label-dependent types~\citep{10.1145/3485485}
% \Citet{Eremondi:2019:ANG:3352468.3341692} presented GDTL,
% a foundational calculus with full dependent types.
% GDTL features
% decidable type checking,
% and allowed the imprecise term $\gqm$ to replace any part of a term or type.
% However, the termination argument for GDTL does not support inductive types.
% The most comprehensive development of gradual dependent types is GCIC~\citep{bertrand:gcic},
% which extends the Calculus of Inductive Constructions (CIC) with gradual types.
% GCIC comes in three, each of which satisfies two of the following properties:
% (1) decidable type checking, (2) the gradual guarantees, and (3) conservatively extending CIC.
% \je{TODO: mention equality if either paper gets accepted}
% The authors show that, to a degree, the compromise of GCIC is necessary:
% they establish a ``fire-triangle'' theorem that shows no dependently typed language can satisfy
% strong normalization, conservative extension of CIC, and the Embedding-Projection Pairs (EP-pairs) property,
% a strengthening of the gradual guarantees.


% In this paper, we take the view that the fire triangle properties are means to an end,
% rather than ends unto themselves. Conservatively extending CIC is a useful trait on its own,
% since it ensures that no ill-typed static programs are allowed in a gradual language.
% Strong normalization, conversely, is mainly useful to
% prove that type-checking is decidable, as well as to establish logical consistency.
% Likewise, EP-pairs are mainly useful for showing the gradual guarantees, and showing that
% gradual programs to not unnecessarily produce $\gqm$ as their result.

We address both these shortcomings in \lang, a \underline{Gr}adual language with \underline{Ind}uctive types.
\lang sacrifices
strong normalization and EP-pairs, but keeps the properties that we actually want: decidable type checking,
(weak) consistency and canonicity, the gradual guarantees, and conservatively extending CIC.
Because type checking is decidable, \lang can be translated into the core calculi of existing dependently typed compilers.
Moreover, we show that \lang does not violate static reasoning principles: propositionally-equal
CIC terms embedded in \lang are observationally-equivalent,
and casts only change the error-behavior of terms, not the concrete results produced.
% and while we lack full EP-pairs,
% we show that casts only change the behaviour of terms relative to dynamic type errors:
% applying casts may produce a term that raises an error in more or less situations, but when a concrete
% result is produced, it is the same result.
% \je{TODO explain EP pairs}

Our main contribution is a translation from \lang to a static type theory:
\begin{itemize}
          \item For implementation, the translation means that existing normalizers
                and code generators can be used ``off-the-shelf'' to compile \lang programs;
  \item For metatheory, the translation serves as a syntactic model in the style of \citet{10.1145/3018610.3018620}, which we use to prove the gradual guarantees
        and other metatheoretic properties;
  \item To enable decidable type checking, we adapt approximate normalization
        from \citet{Eremondi:2019:ANG:3352468.3341692} to a cast calculus, using the syntactic
        model to prove termination;
  \item To model run-time semantics, we translate to \textit{guarded type theory}~\citep{TODO},
        whose non-positive recursive types allow the non-termination of gradual types to be
        exactly represented in a consistent target language;
  \item  Our translation and the theorems about it have been mechanized in Guarded Cubical Agda~\citep{TODO}
\end{itemize}

\section{Two Problems, One Solution}

Our work attacks two main problems that have a common solution.
First, we want to allow \lang to be implemented
by translating them to static dependent types without needing to add features to the static target langauge,
so that existing technology can be
used when compiling them.
The challenge of this is accommodating the non-termination of gradual typing,
since dependently typed core languages typically forbid or restrict non-terminating function definitions.
Second, we want to prove properties about \lang, namely that approximate normalization
terminates (for decidable type checking) and that the gradual guarantees are satisfied.
Unlike the approach of \citet{Eremondi:2019:ANG:3352468.3341692}, the syntactic model
approach scales to handle inductive types, as well as logical-relation
style proofs~\citep{10.1017/S0956796812000056}.

In this section, we explain these two problems, the challenges in solving
them, and a birds-eye view of our approach to solving them.




\subsection{Translation to Support Implementation}

\subsubsection{Don't Reinvent The Wheel}

\subsubsection{An Implementation Strategy}

\subsubsection{The Idris Model of Non-Termination}

\subsubsection{Translating Approximate Normalization}


\subsection{Metatheory}

\subsubsection{Extinguishing the Fire Triangle}

\subsubsection{Static Reasoning in Gradual Code}

\subsection{Modelling Gradual Dependent Types}


\subsubsection{Modelling Approximate Normalization}

\subsubsection{Guarded Type Theory}

\subsubsection{Relating Approximate and Exact Normalization}



\bibliographystyle{ACM-Reference-Format}
\bibliography{myRefs}

% \clearpage
% \input{appendix}

\end{document}
\endinput
%%
%% End of file `sample-acmsmall.tex'.
